\documentclass[%
a4paper,
%twoside,
11pt
]{article}

% encoding, font, language
\usepackage[utf8]{inputenc}
\usepackage[T1]{fontenc}
\usepackage{lmodern}
\usepackage[french]{babel}
% Usage de police Times 
\renewcommand{\rmdefault}{ptm}

\usepackage{nicefrac}

\usepackage[nowarnings]{xcookybooky}

\IfLanguagePatterns{french}
{% French : needed due to incompatibility latin1 of xcookybooky and utf8 needed
    \setHeadlines
    {% translation
        inghead = Ingrédients,
        prephead = Préparation,
        hinthead = Cooky les bons tuyaux,
        continuationhead = Suite,
        continuationfoot = Suite page suivante,
        portionvalue = Portions,
        calory = Valeur calorifique,
    }
}{}

\DeclareRobustCommand{\textcelcius}{\ensuremath{^{\circ}\mathrm{C}}}

\setcounter{secnumdepth}{1}
\renewcommand*{\recipesection}[2][]
{%
    \subsection[#1]{#2}
}
\renewcommand{\subsectionmark}[1]
{% no implementation to display the section name instead
}


\usepackage{hyperref}    % must be the last package
\hypersetup{%
    pdfauthor            = {Ben Viala},
    pdftitle             = {Le CookBook},
    pdfsubject           = {Recettes},
    pdfkeywords          = {recettes, lecookbook, livre de cuisine, cuisine, cookbook},
    pdfstartview         = {FitV},
    pdfview              = {FitH},
    pdfpagemode          = {UseNone}, % Options; UseNone, UseOutlines
    bookmarksopen        = {true},
    pdfpagetransition    = {Glitter},
    colorlinks           = {true},
    linkcolor            = {black},
    urlcolor             = {blue},
    citecolor            = {black},
    filecolor            = {black},
}

\hbadness=10000	% Ignore underfull boxes

\begin{document}

\title{Examples for using \textbf{xcookybooky}}
\title{Le CookBook}
\author{Ben Viala\\ \href{mailto:ben@viala.tech}{ben@viala.tech}}
\maketitle

\begin{abstract}
	Cet ouvrage est mon livre de cuisine. Sobrement intitulé "LE" CookBook, c'est un recueil de mes recettes favorites, ou signatures. Je le partage en espérant qu'il inspirera quelques gourmands, ou motivera des néo-cuisiniers à se lancer.

	Etant français originaire du Sud-Ouest, il regroupe majoritairement des recettes d'origine française ou européenne. La plupart ne conviennent pas à une alimentation végane, ou sans-alcool. 

	Le CookBook a été écrit en langage \texttt{LaTeX}. La génération du CookBook nécessite au
moins la version~1.4 du paquet \texttt{xcookybooky}\footnote{\url{http://www.ctan.org/pkg/xcookybooky}}. Plus d'informations sur la page \texttt{github}\footnote{\url{https://github.com/bennychoubros/LeCookBook}} du projet. 
\end{abstract}

\tableofcontents

\vspace{5em}

\section{Recettes}
L'intégralité des recettes suivantes a été collecté au fil des années. Recettes traditionnelles,
familiales, trouvées sur internet ou bien obtenues par bouche à oreilles, elles sont d'origine
diverses. Cependant elles ont toutes été cuisinées (et goûtées !) par mes soins.
J'ai essayé de retracer les auteurs quand c'était possible.

% background graphic
% Ajout d'une photo en fond si nécessaire
%\setBackgroundPicture[x, y=-2cm, width=\paperwidth-4cm, height, orientation = pagecenter]
%{img/background.jpg}

\begin{recipe}
[%
	preparationtime = {\unit[20]{min}}, %h, min, ...
	bakingtime={\unit[30]{min}}, %h, min, ...
	bakingtemperature={\protect\bakingtemperature{
		topbottomheat=\unit[150]{\textcelsius}}
	},
	portion = {\portion{3}},
	source = {Bennychou}
]
{Crème brûlée}

	%    \graph
	%    {% pictures
	%        small=pic/glass,     % small picture
	%        big=pic/ingredients  % big picture
	%     }

	\introduction{%
		Il est plus aisé de prévoir cette recette la veille, pour laisser la crème reposer au frigo suffisament longtemps.
	}

	\ingredients{%
		3 jaunes d'œufs\\
		30g de sucre en poudre\\
		25cl de crème entière liquide\\
		10cl de lait entier\\
		Sucre de canne roux, ou cassonade\\
		1 dosette de vanille liquide\\
	}

	\preparation{%
		\begin{enumerate}
			\item Préchauffez le four à 150{\textcelsius}.
			\item Mélangez le sucre, la crème, le lait, les jaunes d’œufs et la vanille jusqu'à ce que ce soit parfaitement homogène.
			\item Versez dans des ramequins et cuire 30 min au four.
			\item Laissez refroidir. Les réservez au frigo jusqu'au lendemain.
			\item Juste avant de servir, saupoudrez de sucre et faites caraméliser les crèmes au chalumeau.
		\end{enumerate}
	}

	\hint{%
		Pour faire une crème catalane, remplacer la vanille par un zeste de citron et une
pointe de cannelle. On peut aussi la parfumer avec un peu de graines d'anis vert pilées, c'est encore meilleur. 
	}

\end{recipe}

\begin{recipe}
[%
	preparationtime = {\unit[20]{min}}, %h, min, ...
	bakingtime={\unit[30]{min}}, %h, min, ...
	bakingtemperature={\protect\bakingtemperature{
		topbottomheat=\unit[150]{\textcelsius}}
	},
	portion = {\portion{4}},
	source = {Bennychou}
]
{Crème brûlée au foie gras}

	%    \graph
	%    {% pictures
	%        small=pic/glass,     % small picture
	%        big=pic/ingredients  % big picture
	%     }

	\introduction{%
		Il est plus aisé de prévoir cette recette la veille, pour laisser la crème reposer au frigo suffisament longtemps.
	}

	\ingredients{%
		200g de foie gras mi-cuit\\
		20cl Lait\\
		20cl Crème entière Liquide\\
		4 jaunes d'œufs\\
		10cl de Muscat\\
		Sucre de canne roux, ou cassonade\\
		1 cuillère à café de muscade\\
		poivre\\
		sel\\
	}

	\preparation{%
		\begin{enumerate}
			\item Préchauffez le four à 150{\textcelsius}. 
			\item Coupez le foie gras en morceaux grossiers.
			\item Mixez le foie gras avec le muscat, la crème, le lait, les jaunes d’œufs, sel, poivre et muscade. Mélangez jusqu'à ce que ce soit parfaitement homogène. 
			\item Versez dans les ramequins et cuire 30 min au four.
			\item Laissez refroidir. Les réservez au frigo jusqu'au lendemain. 
			\item Juste avant de servir, saupoudrez de sucre et faites caraméliser les crèmes au chalumeau. 
		\end{enumerate}
	}

	\suggestion[Hors-d'Œuvre]
	{%
		Recette parfaite pour une entrée ou un dessert pour les adeptes du salé, c'est auussi une très bonne mise en bouche. N'hésitez pas à préparer des petites portions individuelles, dans des cuillères allant au four ou des mini-ramequins.
	}

	\hint{%
		S'accompagne bien d'un vin blanc moelleux : Gewurtzraminer ou Coteaux du Layon par exemple. Et pourquoi pas essayer un Saint Georges - Saint-Emilion si c'est le plat principal. 
	}

\end{recipe}

\begin{recipe}
[%
	preparationtime = {\unit[30]{min}}, %h, min, ...
	bakingtime={\unit[1]{h}}, %h, min, ...
	bakingtemperature={\protect\bakingtemperature{
		topbottomheat=\unit[150]{\textcelsius}}
		%fanoven=\unit[230]{\textcelcius},
		%topbottomheat=\unit[195]{\textcelcius},
		%topheat=\unit[195]{\textcelcius},
		%gasstove=Level 2}
	},
	portion = {\portion{4}},
	%calory={\unit[3]{kJ}},
	source = {M. Viala-Vincent}
]
{Tourte aux Pommes de Terre}

	%    \graph
	%    {% pictures
	%        small=pic/glass,     % small picture
	%        big=pic/ingredients  % big picture
	%     }

	\ingredients{%
		2 rouleaux de pâte feuilletée\\
		500g de pommes de terre\\
		6 tranches de poitrine ou lard\\
		100g de crème fraîche\\
		30g de beurre\\
		1 jaune d'æuf\\
		ciboulette\\
		poivre\\
		sel\\
	}

	\preparation{%
		\begin{enumerate}
			\item Disposez un cercle de pâte dans un plat à tarte allant au four.
			\item Préchauffer le four à 150{\textcelsius}
			\item Epluchez les pommes de terre et coupez-les en rondelles fines. Plongez les 10 mn dans l'eau bouillante et égouttez-les.
			\item Disposez les ensuite sur le fond de tarte et recouvrir de tranches de lard.
			\item Poivrez, salez, parsemez de beurre. Recouvrez avec le second cercle de pâte en soudant les bords de la tourte. Badigeonnez de jaune d'æuf et faites cuire pendant 1 h.
			\item Hachez finement la ciboulette. Mélangez dans un bol avec la crème fraîche, sel et poivre.
			\item Sortez la tourte du four. Découpez délicatement le tour du couvercle pour l'ouvrir. Etalez la crème à la ciboulette. Remettre au four pour 10 minutes maximum.
		\end{enumerate}
	}

	\hint{%
		Plat d'hiver complet et lourd ! Parfait avec une salade.
		Pour un accord mets vin, privilégiez un vin rouge du Sud-Ouest type Fronton ou Languedoc Saint Christol, ou un Beaujolais Villages pour ceux qui préfèrent le blanc.
	}

\end{recipe}

\end{document} 
