\documentclass[%
a4paper,
%twoside,
11pt
]{article}

% encoding, font, language
\usepackage[utf8]{inputenc}
\usepackage[T1]{fontenc}
\usepackage{lmodern}
\usepackage[french]{babel}
% Usage de police Times 
\renewcommand{\rmdefault}{ptm}

\usepackage{nicefrac}

\usepackage[nowarnings]{xcookybooky}

\IfLanguagePatterns{french}
{% French : needed due to incompatibility latin1 of xcookybooky and utf8 needed
    \setHeadlines
    {% translation
        inghead = Ingrédients,
        prephead = Préparation,
        hinthead = Cooky les bons tuyaux,
        continuationhead = Suite,
        continuationfoot = Suite page suivante,
        portionvalue = Portions,
        calory = Valeur calorifique,
    }
}{}

\DeclareRobustCommand{\textcelcius}{\ensuremath{^{\circ}\mathrm{C}}}

\setcounter{secnumdepth}{1}
\renewcommand*{\recipesection}[2][]
{%
    \subsection[#1]{#2}
}
\renewcommand{\subsectionmark}[1]
{% no implementation to display the section name instead
}


\usepackage{hyperref}    % must be the last package
\hypersetup{%
    pdfauthor            = {Ben Viala},
    pdftitle             = {Le CookBook},
    pdfsubject           = {Recettes},
    pdfkeywords          = {recettes, lecookbook, livre de cuisine, cuisine, cookbook},
    pdfstartview         = {FitV},
    pdfview              = {FitH},
    pdfpagemode          = {UseNone}, % Options; UseNone, UseOutlines
    bookmarksopen        = {true},
    pdfpagetransition    = {Glitter},
    colorlinks           = {true},
    linkcolor            = {black},
    urlcolor             = {blue},
    citecolor            = {black},
    filecolor            = {black},
}

\hbadness=10000	% Ignore underfull boxes

\begin{document}

\title{Examples for using \textbf{xcookybooky}}
\title{Le CookBook}
\author{Ben Viala\\ \href{mailto:ben@viala.tech}{ben@viala.tech}}
\maketitle

\begin{abstract}
	Cet ouvrage est mon livre de cuisine. Sobrement intitulé "LE" CookBook, c'est un recueil de mes recettes favorites, ou signatures. Je le partage en espérant qu'il inspirera quelques gourmands, ou motivera des néo-cuisiniers à se lancer.

	Etant français originaire du Sud-Ouest, il regroupe majoritairement des recettes d'origine française ou européenne. La plupart ne conviennent pas à une alimentation végane, ou sans-alcool. 

	Le CookBook a été écrit en langage \texttt{LaTeX}. La génération du CookBook nécessite au
moins la version~1.4 du paquet \texttt{xcookybooky}\footnote{\url{http://www.ctan.org/pkg/xcookybooky}}. Plus d'informations sur la page \texttt{github}\footnote{\url{https://github.com/bennychoubros/LeCookBook}} du projet. 
\end{abstract}

\tableofcontents

\vspace{5em}

\section{Recettes}
L'intégralité des recettes suivantes a été collecté au fil des années. Recettes traditionnelles,
familiales, trouvées sur internet ou bien obtenues par bouche à oreilles, elles sont d'origine
diverses. Cependant elles ont toutes été cuisinées (et goûtées !) par mes soins.
J'ai essayé de retracer les auteurs quand c'était possible.

% background graphic
% Ajout d'une photo en fond si nécessaire
%\setBackgroundPicture[x, y=-2cm, width=\paperwidth-4cm, height, orientation = pagecenter]
%{img/background.jpg}

\include{recettes/Bœuf_and_Guinness_stew}
\begin{recipe}
[%
	preparationtime = {\unit[25]{min}}, %h, min, ...
	bakingtime={\unit[10]{min}}, %h, min, ...
	portion = {\portion{2}},
]
{Boudin aux pommes}

	\introduction{%
		%
	}

	\ingredients{%
		% Example from xcookybooky documentation :
		200g de boudin noir\\
		2 pommes\\
		1 citron\\
		30g de beurre\\
		1 pincee de cannelle en poudre\\
		poivre\\
		sel\\
	}

	\preparation{%
		\begin{enumerate}
			\item Pressez le citron et réservez le jus.
			\item Pelez les pommes et coupez-les en 4. Arrosez les de jus de citron. Couper les quartiers en tranches fines et réservez les pommes dans le jus de citron.
			\item Dans une poêle chaude, faites fondre la moitié du beurre. Faites-y revenir les pommes en couvrant, pour éviter qu'elles s'assèchent. Laissez dorer pendant 5 minutes en retournant régulièrement. Saupoudrez de cannelle.
			\item Ajoutez les boudins noirs entiers dans la poêle en les plaçant autour des tranches de pommes. Faites cuire 5 min supplémentaires à feu doux. Salez, poivrez.
		\end{enumerate}
	}

	\suggestion{%
		Servi avec une purée de pommes de terre ou de butternut, c'est encore meilleur.
		La recette est avec un boudin nature, mais fonctionne trèss bien avec du boudin aux pommes, à l'oignon, aux châtaignes, etc...
	}

	\hint{%
		S'accommode très bien d'un Côtes de Bourg ou d'un Côtes du Roussillon, ou d'un Bergerac. Mention très bien avec un Gevrey-Chambertin...
	}

\end{recipe}

\begin{recipe}
[%
	preparationtime = {\unit[20]{min}}, %h, min, ...
	bakingtime={\unit[30]{min}}, %h, min, ...
	bakingtemperature={\protect\bakingtemperature{
		topbottomheat=\unit[150]{\textcelsius}}
	},
	portion = {\portion{3}},
	source = {Bennychou}
]
{Crème brûlée}

	%    \graph
	%    {% pictures
	%        small=pic/glass,     % small picture
	%        big=pic/ingredients  % big picture
	%     }

	\introduction{%
		Il est plus aisé de prévoir cette recette la veille, pour laisser la crème reposer au frigo suffisament longtemps.
	}

	\ingredients{%
		3 jaunes d'œufs\\
		30g de sucre en poudre\\
		25cl de crème entière liquide\\
		10cl de lait entier\\
		Sucre de canne roux, ou cassonade\\
		1 dosette de vanille liquide\\
	}

	\preparation{%
		\begin{enumerate}
			\item Préchauffez le four à 150{\textcelsius}.
			\item Mélangez le sucre, la crème, le lait, les jaunes d’œufs et la vanille jusqu'à ce que ce soit parfaitement homogène.
			\item Versez dans des ramequins et cuire 30 min au four.
			\item Laissez refroidir. Les réservez au frigo jusqu'au lendemain.
			\item Juste avant de servir, saupoudrez de sucre et faites caraméliser les crèmes au chalumeau.
		\end{enumerate}
	}

	\hint{%
		Pour faire une crème catalane, remplacer la vanille par un zeste de citron et une
pointe de cannelle. On peut aussi la parfumer avec un peu de graines d'anis vert pilées, c'est encore meilleur. 
	}

\end{recipe}

\begin{recipe}
[%
	preparationtime = {\unit[20]{min}}, %h, min, ...
	bakingtime={\unit[30]{min}}, %h, min, ...
	bakingtemperature={\protect\bakingtemperature{
		topbottomheat=\unit[150]{\textcelsius}}
	},
	portion = {\portion{4}},
	source = {Bennychou}
]
{Crème brûlée au foie gras}

	%    \graph
	%    {% pictures
	%        small=pic/glass,     % small picture
	%        big=pic/ingredients  % big picture
	%     }

	\introduction{%
		Il est plus aisé de prévoir cette recette la veille, pour laisser la crème reposer au frigo suffisament longtemps.
	}

	\ingredients{%
		200g de foie gras mi-cuit\\
		20cl Lait\\
		20cl Crème entière Liquide\\
		4 jaunes d'œufs\\
		10cl de Muscat\\
		Sucre de canne roux, ou cassonade\\
		1 cuillère à café de muscade\\
		poivre\\
		sel\\
	}

	\preparation{%
		\begin{enumerate}
			\item Préchauffez le four à 150{\textcelsius}. 
			\item Coupez le foie gras en morceaux grossiers.
			\item Mixez le foie gras avec le muscat, la crème, le lait, les jaunes d’œufs, sel, poivre et muscade. Mélangez jusqu'à ce que ce soit parfaitement homogène. 
			\item Versez dans les ramequins et cuire 30 min au four.
			\item Laissez refroidir. Les réservez au frigo jusqu'au lendemain. 
			\item Juste avant de servir, saupoudrez de sucre et faites caraméliser les crèmes au chalumeau. 
		\end{enumerate}
	}

	\suggestion[Hors-d'Œuvre]
	{%
		Recette parfaite pour une entrée ou un dessert pour les adeptes du salé, c'est auussi une très bonne mise en bouche. N'hésitez pas à préparer des petites portions individuelles, dans des cuillères allant au four ou des mini-ramequins.
	}

	\hint{%
		S'accompagne bien d'un vin blanc moelleux : Gewurtzraminer ou Coteaux du Layon par exemple. Et pourquoi pas essayer un Saint Georges - Saint-Emilion si c'est le plat principal. 
	}

\end{recipe}

\begin{recipe}
[%
	preparationtime = {\unit[30]{min}}, %h, min, ...
	bakingtime={\unit[30]{min}}, %h, min, ...
	portion = {\portion{4}},
]
{Colcannon mash}

	%    \graph
	%    {% pictures
	%        small=pic/glass,     % small picture
	%        big=pic/ingredients  % big picture
	%     }


	\ingredients{%
		1kg de pommes de terre\\
		250g de chou\\
		Oignons nouveaux\\
		50g de beurre\\
		10cl de lait entier\\
		Poivre\\
		Sel\\
	}

	\preparation{%
		\begin{enumerate}
			\item Epluchez et couper les pommes de terres en dés.
			\item Emincez finement le chou et les oignons nouveaux.
			\item Mettez les dés de pommes de terre dans une cocotte d’eau froide et portez à ébullition. Laissez mijotez jusqu'à ce qu'elles soient cuites.
			\item Pendant ce temps, faites cuire le chou à la vapeur (panier cuisson ou à l'eau dans une cocotte) jusqu'à ce qu'il soit tendre.
			\item Quand les pommes sont cuites, passez les en purée en ajoutant la moitié du beurre et le lait.
			\item Ajoutez les oignons et le chou. Mélangez-bien.
			\item Salez. Poivrez. Servez avec le reste du beurre.
		\end{enumerate}
	}

	\hint{%
		Salon la saison, on peut remplacer par de la ciboulette, pour réaliser du \texttt{Champ}.
		Ce plat typique irlandais accompagnera à merveille Irish stew, Beef\&Guinness ou autre plat traditionnel...
	}

\end{recipe}

\begin{recipe}
[%
	preparationtime = {\unit[10]{min}}, %h, min, ...
	bakingtime={sans cuisson}, %h, min, ...
	portion = {\portion{2}},
	source = {T. Clouet}
]
{Hareng pomme à l’huile}

	%    \graph
	%    {% pictures
	%        small=pic/glass,     % small picture
	%        big=pic/ingredients  % big picture
	%     }

	\ingredients{%
		2 filets de harengs fumés\\
		4 pommes de terres nouvelles (type charlotte)\\
		1 échalote\\
		2 oignons nouveaux\\
		3 cuillères à soupe d'huile\\
		1 cuillère à soupe de vinaigre de vin rouge\\
		poivre\\
		sel\\
	}

	\preparation{%
		\begin{enumerate}
			\item Pelez les pommes de terre. Faites-les cuire dans une casserole remplie d’eau froide salée pendant une trentaine de minutes. Vérifiez la cuisson à l’aide de la pointe d’un couteau. Egouttez-les. Coupez-les en tranches.
			\item Otez la première peau de votre échalote et ciselez-la.
			\item Otez la première peau de vos oignons nouveaux puis coupez-les en rondelles. Séparez les anneaux.
			\item Dans un bol, mélangez le vinaigre avec du sel et du poivre puis ajoutez l’huile ainsi que l’échalote.
			\item Servez vos filets de harengs en les accompagnant de tranches de pommes de terre. Arrosez l’ensemble de vinaigrette. Agrémentez d’anneaux d’oignons nouveaux.
		\end{enumerate}
	}


	\hint{%
		Vous pouvez ajouter un peu de cerfeuil ou du persil plat à votre vinaigrette.
	}

\end{recipe}

\include{recettes/Magret_de_canard_rôti_et_jus_de_sangria}
\include{recettes/Oeufs_brouillés}
\begin{recipe}
[%
	preparationtime = {\unit[10]{min}}, %h, min, ...
	portion = {\portion{2}},
]
{Pisco sour}

	%    \graph
	%    {% pictures
	%        small=pic/glass,     % small picture
	%        big=pic/ingredients  % big picture
	%     }


	\ingredients{%
		10cl de Pisco\\
		1 citron\\
		2cl de sucre\\
		1 blanc d'œuf\\
		4-6 glaçons\\
		quelques gouttes d'Angustura\\
	}

	\preparation{%
		\begin{enumerate}
			\item Pressez le citron pour obtenir environ 10cl de jus.
			\item Pilez les glaçons.
			\item Versez dans un shaker le jus de citron, le sucre puis le Pisco. Ajoutez le blanc d'oeuf et de la glace pilée. 
			\item Shake it, baby.
			\item Versez dans un verre. Ajoutez quelques gouttes d'Angostura sur la mousse.
		\end{enumerate}
	}

	\hint{%
		Pour ceux qui n'aiment pas le bitter, on peut remplacer l'Angostura par un peu de cannelle en poudre.
	}

\end{recipe}

\begin{recipe}
[%
	preparationtime = {\unit[]{min}}, %h, min, ...
	bakingtime={\unit[]{h}}, %h, min, ...
	bakingtemperature={\protect\bakingtemperature{
		topbottomheat=\unit[]{\textcelsius}}
		%fanoven=\unit[230]{\textcelcius},
		%topbottomheat=\unit[195]{\textcelcius},
		%topheat=\unit[195]{\textcelcius},
		%gasstove=Level 2}
	},
	portion = {\portion{4}},
	source = {Bennychou}
]
{Poêlée de chou aux châtaignes}

	%    \graph
	%    {% pictures
	%        small=pic/glass,     % small picture
	%        big=pic/ingredients  % big picture
	%     }

	\ingredients{%
		1/2 chou\\
		500g de châtaignes pelées\\
		750g de pommes de terre\\
		2 échalotes\\
		30g de beurre\\
		poivre\\
		sel\\
	}

	\preparation{%
		\begin{enumerate}
			\item Pelez et coupez les pommes de terres en dés.
			\item Faites chauffer le beurre dans une grande poêle ou une sauteuse.. Ajoutez les pommes de terres, en remuant régulièrement pour éviter qu'elles attachent au fond.
			\item Découper le chou en lanières. Les ajouter dans la poêle après environ 10min de cuisson des pommes de terre. Ajouter un demi-verre d'eau.
			\item Pendant la cuisson, émincer les échalotes.
			\item Les ajouter 5 à 7 mn plus tard, avec les châtaignes. Ajoutez à nouveau un peu d'eau si nécessaire. Salez, poivrez.
			\item Remuer régulièrement jusqu'á la cuisson complète des pommes de terres. Servez chaud.
		\end{enumerate}
	}

	\suggestion{%
		Un plat qui sent bon l'automne. On peut aussi rajouter des lardons pour les gourmands non-végétariens !
		L'eau peut être remplacée par du bouillon de légumes, pour plus de parfum.
	}

	\hint{%
		Ce plat s'accompagne très bien d'un vin rouge léger : cherchez du côté des vins de Loire ou du Beaujolais.
	}

\end{recipe}

\begin{recipe}
[%
	preparationtime = {\unit[30]{min}}, %h, min, ...
	bakingtime={\unit[40]{min}}, %h, min, ...
	portion = {\portion{2}},
]
{Saint-Jacques à la fondue de poireaux}

	%    \graph
	%    {% pictures
	%        small=pic/glass,     % small picture
	%        big=pic/ingredients  % big picture
	%     }

	\ingredients{%
		6-8 Noix de Saint Jacques\\
		2 Blancs de poireaux\\
		30g de beurre\\
		1/2 citron\\
		2 cuillères à soupe de crème fraîche\\
		1 cuillère à soupe de moutarde\\
		poivre\\
		sel\\
	}

	\preparation{%
		\begin{enumerate}
			\item Lavez-les poireaux. Coupez-les en deux puis en petits morceaux.
			\item Faites fondre le beurre dans une poêle. Ajoutez les poireaux, mélangez, couvrez et faites cuire à feu très doux en remuant pendant 25 min environ.
			\item Sortir du frigo les noix de Saint Jacque et laissez-les reposer à température ambiante.
			\item Pressez le citron.
			\item Quand les poireaux sont prêts, ajoutez le jus de citron, la crème et la moutarde. Poivrez, salez et mélangez bien. Couvrez et laissez cuire encore 10 min à feu doux.
			\item Faire fondre une noix de beurre dans une poêle. Y cuire les Saint
				Jacques à feu vif : entre 2 à 3 minutes par face maximum.
			\item Assaisonez à votre goût et servir chaud.
		\end{enumerate}
	}

	\hint{%
		Avec un peu de poivre à huître sur les noix, c'est en core meilleur.
		Recette sublimée si accompagnée d'un Chablis. Ou plus sobrement d'un Muscadet.
	}

\end{recipe}

\begin{recipe}
[%
	preparationtime = {\unit[10]{min}}, %h, min, ...
	bakingtime={\unit[30]{min}}, %h, min, ...
	portion = {\portion{2}},
	source = {Bennychou}
]
{Salade de lentilles}

	%    \graph
	%    {% pictures
	%        small=pic/glass,     % small picture
	%        big=pic/ingredients  % big picture
	%     }

	\ingredients{%
		140g de lentilles vertes\\
		2 échalotes\\
		3 cuillères à soupe de vinaigre balsamique\\	
		3 cuillères à soupe d'huile\\
		poivre\\
		sel\\
	}

	\preparation{%
		\begin{enumerate}
			\item Faire cuire les lentilles environ 25 min dans l'eau froide à partir du bouillonnement de l'eau. Les égoutter et les laisser refroidir.
			\item Emincer finement les échalotes. 
			\item Préparez la vinaigrette, en mélangeant vinaigre, huile et échalotes. Salez et poivrez selon votre goût.
			\item Quand les lentilles sont froides, mélangez avec la vinaigrette et servir.
		\end{enumerate}
	}

	\hint{%
		Un classique : Ajout de lardons, changez la vinaigrette selon les goûts, ajout de tomates et feta... Faites-vous plaisir.
	}

\end{recipe}

\begin{recipe}
[%
	preparationtime = {\unit[15]{min}}, %h, min, ...
	bakingtime={\unit[2]{h}}, %h, min, ...
	bakingtemperature={\protect\bakingtemperature{
		topbottomheat=\unit[175]{\textcelsius}}
	},
	portion = {\portion{2}},
	source = {Bennychou}
]
{Souris d'agneau confite au miel}

	%    \graph
	%    {% pictures
	%        small=pic/glass,     % small picture
	%        big=pic/ingredients  % big picture
	%     }

	\ingredients{%
	 	2 souris d'agneau\\
		1 tête d'ail\\
		5 cuillères à soupe d'huile d'olive\\
		5 branches de thym\\
		2 cuillères à soupe de miel de fleurs\\
		Poivre\\
		Sel\\
	}

	\preparation{%
		\begin{enumerate}
			\item Dans une casserole faites fondre le miel dans l'huile d'olive jusqu'à l'obtention d'un mélange homogène. 
			\item Préchauffez le four à 175{\textcelsius}.
			\item Nappez les souris d'agneau  du mélange huile d'olive-miel.	
			\item Disposez les souris dans un plat couvert allant au four. Placez autour toutes les gousses d'ail non épluchées et le thym. Salez, poivrez.
			\item Enfournez pendant environ 2 bonnes à 175{\textcelsius}.
			\item A partir d'une heure de cuisson, ajoutez de temps en temps un peu d'eau ou de bouillon de légumes pour évitez que la viande sèchee. Retournez la viande de temps en temps.
		\end{enumerate}
	}

	\hint{%
		Très bon accompagné d'un Faugères, d'un Pic Saint Loup ou d'un Côtes du Rhone.
	}

\end{recipe}

\begin{recipe}
[%
	preparationtime = {\unit[30]{min}}, %h, min, ...
	bakingtime={\unit[1]{h}}, %h, min, ...
	bakingtemperature={\protect\bakingtemperature{
		topbottomheat=\unit[150]{\textcelsius}}
		%fanoven=\unit[230]{\textcelcius},
		%topbottomheat=\unit[195]{\textcelcius},
		%topheat=\unit[195]{\textcelcius},
		%gasstove=Level 2}
	},
	portion = {\portion{4}},
	%calory={\unit[3]{kJ}},
	source = {M. Viala-Vincent}
]
{Tourte aux Pommes de Terre}

	%    \graph
	%    {% pictures
	%        small=pic/glass,     % small picture
	%        big=pic/ingredients  % big picture
	%     }

	\ingredients{%
		2 rouleaux de pâte feuilletée\\
		500g de pommes de terre\\
		6 tranches de poitrine ou lard\\
		100g de crème fraîche\\
		30g de beurre\\
		1 jaune d'æuf\\
		ciboulette\\
		poivre\\
		sel\\
	}

	\preparation{%
		\begin{enumerate}
			\item Disposez un cercle de pâte dans un plat à tarte allant au four.
			\item Préchauffer le four à 150{\textcelsius}
			\item Epluchez les pommes de terre et coupez-les en rondelles fines. Plongez les 10 mn dans l'eau bouillante et égouttez-les.
			\item Disposez les ensuite sur le fond de tarte et recouvrir de tranches de lard.
			\item Poivrez, salez, parsemez de beurre. Recouvrez avec le second cercle de pâte en soudant les bords de la tourte. Badigeonnez de jaune d'æuf et faites cuire pendant 1 h.
			\item Hachez finement la ciboulette. Mélangez dans un bol avec la crème fraîche, sel et poivre.
			\item Sortez la tourte du four. Découpez délicatement le tour du couvercle pour l'ouvrir. Etalez la crème à la ciboulette. Remettre au four pour 10 minutes maximum.
		\end{enumerate}
	}

	\hint{%
		Plat d'hiver complet et lourd ! Parfait avec une salade.
		Pour un accord mets vin, privilégiez un vin rouge du Sud-Ouest type Fronton ou Languedoc Saint Christol, ou un Beaujolais Villages pour ceux qui préfèrent le blanc.
	}

\end{recipe}

\end{document} 
