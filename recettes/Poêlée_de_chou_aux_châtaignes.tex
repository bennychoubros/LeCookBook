\begin{recipe}
[%
	preparationtime = {\unit[15]{min}}, %h, min, ...
	bakingtime={\unit[25]{min}}, %h, min, ...
	portion = {\portion{4}},
	source = {Bennychou}
]
{Poêlée de chou aux châtaignes}

	%    \graph
	%    {% pictures
	%        small=pic/glass,     % small picture
	%        big=pic/ingredients  % big picture
	%     }

	\ingredients{%
		1/2 chou\\
		500g de châtaignes pelées\\
		750g de pommes de terre\\
		2 échalotes\\
		30g de beurre\\
		poivre\\
		sel\\
	}

	\preparation{%
		\begin{enumerate}
			\item Pelez et coupez les pommes de terres en dés.
			\item Faites chauffer le beurre dans une grande poêle ou une sauteuse.. Ajoutez les pommes de terres, en remuant régulièrement pour éviter qu'elles attachent au fond.
			\item Découper le chou en lanières. Les ajouter dans la poêle après environ 10min de cuisson des pommes de terre. Ajouter un demi-verre d'eau.
			\item Pendant la cuisson, émincer les échalotes.
			\item Les ajouter 5 à 7 mn plus tard, avec les châtaignes. Ajoutez à nouveau un peu d'eau si nécessaire. Salez, poivrez.
			\item Remuer régulièrement jusqu'á la cuisson complète des pommes de terres. Servez chaud.
		\end{enumerate}
	}

	\suggestion{%
		Un plat qui sent bon l'automne. On peut aussi rajouter des lardons pour les gourmands non-végétariens !
		L'eau peut être remplacée par du bouillon de légumes, pour plus de parfum.
	}

	\hint{%
		Ce plat s'accompagne très bien d'un vin rouge léger : cherchez du côté des vins de Loire ou du Beaujolais.
	}

\end{recipe}
