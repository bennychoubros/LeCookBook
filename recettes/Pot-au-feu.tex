\begin{recipe}
[%
	preparationtime = {\unit[30]{min}}, %h, min, ...
	bakingtime={\unit[3]{h}}, %h, min, ...
	portion = {\portion{2}},
]
{Pot-au-feu}

	%    \graph
	%    {% pictures
	%        small=pic/glass,     % small picture
	%        big=pic/ingredients  % big picture
	%     }

	\introduction{%
		Comme tout bon ragoût, l'idéal est de le préparer la veille et de le réchauffer à feu doux avant dégustation.
	}

	\ingredients{%
		500g de bœuf à braiser (paleron, gîte...)\\
		2 Os à moelle\\
		2 navets\\
		2 carottes\\
		1 poireau\\
		1 oignon\\
		2 clous de girofle\\
		1 gousse d'ail\\
		1 branche de céleri\\
		Thym\\
		Laurier\\
		Persil\\
		Poivre\\
		Sel\\
	}

	\preparation{%
		\begin{enumerate}
			\item Piquer l'oignon entier avec les clous de girofle. 
			\item Eplucher et couper les autres légumes en morceaux.
			\item Dans une cocotte, placer la viande, l'oignon et le bouquet garni
				(thym/laurier/persil) dans 2 litres d'eau froide.
			\item Placer sur le feu à feu doux. Porter à petite ébullition au maximum. Laisser mijoter 2h.
			\item Recouvrir les os à moelle de gros sel.
			\item Au bout de 2h de cuisson, ajouter les os et les carottes. Laisser mijoter.
			\item Apres 15 minutes, ajouter les navets, l'ail et le céleri. Salez,
				poivrez. Laissez mijoter 45 minutes environ.
			\item Retirez du feu et servez.
		\end{enumerate}
	}

	\hint{%
		Un régal avec un Beaujolais, ou un vin du Val de Loire.
	}

\end{recipe}
