\begin{recipe}
[%
	preparationtime = {\unit[45]{min}}, %h, min, ...
	bakingtime={\unit[45]{min}}, %h, min, ...
	portion = {\portion{4}},
	source = {M. Viala}
]
{Truffade}

	%    \graph
	%    {% pictures
	%        small=pic/glass,     % small picture
	%        big=pic/ingredients  % big picture
	%     }

	\ingredients{%
		1kg de pommes de terre\\
		400g de tome fraîche\\
		50g de beurre\\
		4 gousses d'ail\\
		Persil\\
		Poivre\\
		Sel\\
	}

	\preparation{%
		\begin{enumerate}
			\item Eplucher et couper les pommes de terre en rondelles.
			\item Dans une sauteuse, faites cuire les pommes de terre dans le beurre. Remuer régulièrement.
			\item Pendant ce temps, émincer la tome fraîche en tranches fines.
			\item Eplucher et couper l'ail en morceaux grossiers.
			\item Quand les pommes de terre sont cuites à point, ajoutez progressivement la tome émincée.	
			\item Laissez fondre puis Ajoutez l'ail et le persil. Salez et poivrez à votre goût.
			\item Quand le mélange est bien filant, servez chaud !
		\end{enumerate}
	}

	\suggestion{%
		Servez bien chaud, accompagnez d'une salade verte pour un plat végétarien. Ou bien ajoutez des lardons, une tranche de jambon de pays, ou une saucisse d'Auvergne pour ceux qui ont faim !.
	}

	\suggestion[Variantes]
	{%
		N'hésitez pas à changer le fromage selon vos goûts, ou l'origine de la recette :
		Laguiole, Salers, Cantal... Avec ou sans lard... Truffade ou Retortilat... Faites votre choix !
	}
	\hint{%
		Avec un Gaillac, un Cahors, un Fronton... Ou un Marcillac, évidemment.
	}

\end{recipe}
