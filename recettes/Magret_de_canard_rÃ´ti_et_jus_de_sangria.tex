\begin{recipe}
[%
	preparationtime = {\unit[30]{min}}, %h, min, ...
	bakingtime={\unit[30]{min}}, %h, min, ...
	portion = {\portion{2}},
	source = {Chef Damien}
]
{Magret de canard rôti et jus de sangria}

	%    \graph
	%    {% pictures
	%        small=pic/glass,     % small picture
	%        big=pic/ingredients  % big picture
	%     }


	\ingredients{%
		1 magret de canard\\
		1 orange\\
		2 anchois à l'huile\\
		15cl de vin rouge\\
		25 g de sucre de canne roux ou cassonade\\
		3 graines de cardamone\\
		1 baton de cannelle\\
		1 clou de girofle\\
		1 cuillère à café de fécule\\
		piment d'Espelette\\
		sel\\
	}

	\preparation{%
		\begin{enumerate}
			\item Dans une casserole, ajoutez le vin rouge, le sucre, l'orange en morceaux, les épices, le vin et faites réduire à feu très doux.

Déposez le magret coté peau dans une poêle froide. Mettez à chauffer très doucement, pour finalement obtenir (au bout de 20 bonnes minutes à feu très doux) une très fine peau croustillante, et beaucoup de graisse fondue.
			\item Evacuez la graisse pendant la fonte du magret.
Il faut maintenant monter le feu, retournez les magrets et assaisonnez légèrement.
			\item Cuisez coté chair 3 min environ, et débarrassez du feu ce magret pour le laisser se reposer. Hachez les anchois finement et mélangez-les avec du piment d'Espelette.  Mettez cette pâte sur les magrets côté peau.
			\item Lorsque la sauce est bien réduite, passez-la à la passoire fine, remettez la dans la casserole, portez à ébullition et liez légèrement avec la fécule. La sauce doit être nappante sans excés.
Réchauffez rapidement les magrets au four et servez sans attendre. 
		\end{enumerate}
	}

	\suggestion[Suggestion du Chef Damien]
	{%
		Cette sauce est très simple à réaliser, et elle est délicieuse ! Elle se marie très bien avec les viandes rouges, le poulet, mais aussi avec des poissons comme le turbot, le saumon ou la lotte.
	}

	\hint{%
		Pour accompagner, la liste est vaste : un régal avec un Corbières, un Bergerac, un Buzet, un Bordeaux... Faites selon vos goûts !  Et pourquoi pas un vin d'Arbois ou un Gamay Coteaux du Giennois ?
	}

\end{recipe}
