\begin{recipe}
[%
	preparationtime = {\unit[30]{min}}, %h, min, ...
	bakingtime={\unit[1]{h}}, %h, min, ...
	bakingtemperature={\protect\bakingtemperature{
		topbottomheat=\unit[150]{\textcelsius}}
		%fanoven=\unit[230]{\textcelcius},
		%topbottomheat=\unit[195]{\textcelcius},
		%topheat=\unit[195]{\textcelcius},
		%gasstove=Level 2}
	},
	portion = {\portion{4}},
	%calory={\unit[3]{kJ}},
	source = {M. Viala-Vincent}
]
{Tourte aux Pommes de Terre}

	%    \graph
	%    {% pictures
	%        small=pic/glass,     % small picture
	%        big=pic/ingredients  % big picture
	%     }

	\ingredients{%
		2 rouleaux de pâte feuilletée\\
		500g de pommes de terre\\
		6 tranches de poitrine ou lard\\
		100g de crème fraîche\\
		30g de beurre\\
		1 jaune d'æuf\\
		ciboulette\\
		poivre\\
		sel\\
	}

	\preparation{%
		\begin{enumerate}
			\item Disposez un cercle de pâte dans un plat à tarte allant au four.
			\item Préchauffer le four à 150{\textcelsius}
			\item Epluchez les pommes de terre et coupez-les en rondelles fines. Plongez les 10 mn dans l'eau bouillante et égouttez-les.
			\item Disposez les ensuite sur le fond de tarte et recouvrir de tranches de lard.
			\item Poivrez, salez, parsemez de beurre. Recouvrez avec le second cercle de pâte en soudant les bords de la tourte. Badigeonnez de jaune d'æuf et faites cuire pendant 1 h.
			\item Hachez finement la ciboulette. Mélangez dans un bol avec la crème fraîche, sel et poivre.
			\item Sortez la tourte du four. Découpez délicatement le tour du couvercle pour l'ouvrir. Etalez la crème à la ciboulette. Remettre au four pour 10 minutes maximum.
		\end{enumerate}
	}

	\hint{%
		Plat d'hiver complet et lourd ! Parfait avec une salade.
		Pour un accord mets vin, privilégiez un vin rouge du Sud-Ouest type Fronton ou Languedoc Saint Christol, ou un Beaujolais Villages pour ceux qui préfèrent le blanc.
	}

\end{recipe}
