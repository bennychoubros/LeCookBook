\begin{recipe}
[%
	preparationtime = {\unit[20]{min}}, %h, min, ...
	bakingtime={\unit[30]{min}}, %h, min, ...
	bakingtemperature={\protect\bakingtemperature{
		topbottomheat=\unit[150]{\textcelsius}}
	},
	portion = {\portion{3}},
	source = {Bennychou}
]
{Crème brûlée}

	%    \graph
	%    {% pictures
	%        small=pic/glass,     % small picture
	%        big=pic/ingredients  % big picture
	%     }

	\introduction{%
		Il est plus aisé de prévoir cette recette la veille, pour laisser la crème reposer au frigo suffisament longtemps.
	}

	\ingredients{%
		3 jaunes d'œufs\\
		30g de sucre en poudre\\
		25cl de crème entière liquide\\
		10cl de lait entier\\
		Sucre de canne roux, ou cassonade\\
		1 dosette de vanille liquide\\
	}

	\preparation{%
		\begin{enumerate}
			\item Préchauffez le four à 150{\textcelsius}.
			\item Mélangez le sucre, la crème, le lait, les jaunes d’œufs et la vanille jusqu'à ce que ce soit parfaitement homogène.
			\item Versez dans des ramequins et cuire 30 min au four.
			\item Laissez refroidir. Les réservez au frigo jusqu'au lendemain.
			\item Juste avant de servir, saupoudrez de sucre et faites caraméliser les crèmes au chalumeau.
		\end{enumerate}
	}

	\hint{%
		Pour faire une crème catalane, remplacer la vanille par un zeste de citron et une
pointe de cannelle. On peut aussi la parfumer avec un peu de graines d'anis vert pilées, c'est encore meilleur. 
	}

\end{recipe}
