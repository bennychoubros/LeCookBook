\begin{recipe}
[%
	preparationtime = {\unit[25]{min}}, %h, min, ...
	bakingtime={\unit[10]{min}}, %h, min, ...
	portion = {\portion{2}},
]
{Boudin aux pommes}

	\introduction{%
		%
	}

	\ingredients{%
		% Example from xcookybooky documentation :
		200g de boudin noir\\
		2 pommes\\
		1 citron\\
		30g de beurre\\
		1 pincee de cannelle en poudre\\
		poivre\\
		sel\\
	}

	\preparation{%
		\begin{enumerate}
			\item Pressez le citron et réservez le jus.
			\item Pelez les pommes et coupez-les en 4. Arrosez les de jus de citron. Couper les quartiers en tranches fines et réservez les pommes dans le jus de citron.
			\item Dans une poêle chaude, faites fondre la moitié du beurre. Faites-y revenir les pommes en couvrant, pour éviter qu'elles s'assèchent. Laissez dorer pendant 5 minutes en retournant régulièrement. Saupoudrez de cannelle.
			\item Ajoutez les boudins noirs entiers dans la poêle en les plaçant autour des tranches de pommes. Faites cuire 5 min supplémentaires à feu doux. Salez, poivrez.
		\end{enumerate}
	}

	\suggestion{%
		Servi avec une purée de pommes de terre ou de butternut, c'est encore meilleur.
		La recette est avec un boudin nature, mais fonctionne trèss bien avec du boudin aux pommes, à l'oignon, aux châtaignes, etc...
	}

	\hint{%
		S'accommode très bien d'un Côtes de Bourg ou d'un Côtes du Roussillon, ou d'un Bergerac. Mention très bien avec un Gevrey-Chambertin...
	}

\end{recipe}
