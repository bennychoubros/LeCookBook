\begin{recipe}
[%
	preparationtime = {\unit[30]{min}}, %h, min, ...
	bakingtime={\unit[30]{min}}, %h, min, ...
	portion = {\portion{5-6}},
]
{Chipirons à la Basquaise}

	%    \graph
	%    {% pictures
	%        small=pic/glass,     % small picture
	%        big=pic/ingredients  % big picture
	%     }

	\ingredients{%
		1kg de calamars\\
		3 tomates\\
		10cl d'Armagnac\\
		40g de beurre\\
		2 gousses d'ail\\
		1 échalote\\
		1 pincée de piment d'Espelette\\
		Persil\\
		Poivre\\
		Sel\\
	}

	\preparation{%
		\begin{enumerate}
			\item Vider et nettoyer les calamars. Les couper en morceaux et les laisser égoutter.
			\item Couper les tomates en dés. Emincer les échalottes. Couper (ou écraser) l'ail en petits morceaux.
			\item Dans une poêle chaude, faire revenir l'échalote émincée avec un peu de beurre. Ajouter les calamars durant 5 mn.
			\item Vider le jus de la première cuisson, faire revenir avec du beurre l'ail écrasé et les calamars. pour les colorer durant 2~3 minutes.
			\item Verser l'Armagnac et faire flamber.
			\item Une fois flambé, ajouter les tomates et le piment d'Espelette. Laisser mijoter à feu doux.
			\item En fin de cuisson, assaisoner et ajouter le persil.
		\end{enumerate}
	}

	\hint{%
		Restez local : Blanc ou Rouge, accompagnez d'une bouteille d'Irouléguy.
		Accompagnez d'un riz blanc ou de pommes vapeurs. C'est encore meilleur le lendemain, mijoté une deuxième fois.
	}

\end{recipe}
