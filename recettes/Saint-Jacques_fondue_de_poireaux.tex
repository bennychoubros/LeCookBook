\begin{recipe}
[%
	preparationtime = {\unit[30]{min}}, %h, min, ...
	bakingtime={\unit[40]{min}}, %h, min, ...
	portion = {\portion{2}},
]
{Saint-Jacques à la fondue de poireaux}

	%    \graph
	%    {% pictures
	%        small=pic/glass,     % small picture
	%        big=pic/ingredients  % big picture
	%     }

	\ingredients{%
		6-8 Noix de Saint Jacques\\
		2 Blancs de poireaux\\
		30g de beurre\\
		1/2 citron\\
		2 cuillères à soupe de crème fraîche\\
		1 cuillère à soupe de moutarde\\
		poivre\\
		sel\\
	}

	\preparation{%
		\begin{enumerate}
			\item Lavez-les poireaux. Coupez-les en deux puis en petits morceaux.
			\item Faites fondre le beurre dans une poêle. Ajoutez les poireaux, mélangez, couvrez et faites cuire à feu très doux en remuant pendant 25 min environ.
			\item Sortir du frigo les noix de Saint Jacque et laissez-les reposer à température ambiante.
			\item Pressez le citron.
			\item Quand les poireaux sont prêts, ajoutez le jus de citron, la crème et la moutarde. Poivrez, salez et mélangez bien. Couvrez et laissez cuire encore 10 min à feu doux.
			\item Faire fondre une noix de beurre dans une poêle. Y cuire les Saint
				Jacques à feu vif : entre 2 à 3 minutes par face maximum.
			\item Assaisonez à votre goût et servir chaud.
		\end{enumerate}
	}

	\hint{%
		Avec un peu de poivre à huître sur les noix, c'est en core meilleur.
		Recette sublimée si accompagnée d'un Chablis. Ou plus sobrement d'un Muscadet.
	}

\end{recipe}
