\begin{recipe}
[%
	preparationtime = {\unit[1]{h}}, %h, min, ...
	bakingtime={\unit[5]{h}}, %h, min, ...
	portion = {\portion{4}},
]
{Boeuf Bourguignon}

	%    \graph
	%    {% pictures
	%        small=pic/glass,     % small picture
	%        big=pic/ingredients  % big picture
	%     }

	\introduction{%
		Comme tout bon ragoût, l'idéal est de le préparer la veille et de le réchauffer à feu doux avant dégustation.
	}

	\ingredients{%
		600g de bœuf à braiser\\
		(basse-côte, paleron, gîte...)\\
		100g de lardons\\
		100g de champignons\\
		4 carottes\\
		4 oignons\\
		2 carrés de chocolat noir\\
		100g de beurre\\
		50cl de vin rouge\\
		Thym\\
		Laurier\\
		Persil\\
		Poivre\\
		Sel\\
	}

	\preparation{%
		\begin{enumerate}
			\item Couper le bœuf en cubes grossiers.
			\item Eplucher et couper les légumes en morceaux.
			\item Dans une poêle chaude, faire revenir les oignons dans du beurre,
				jusqu'à ce qu'ils soient transparents. Les verser dans une cocotte.
			\item Faire cuire la viande à feu fort dans une poêle avec du beurre, pour les colorer.
			\item Quand elle est bien dorée, ajouter les carottes, les champignons et les lardons  pour les saisir légèrement. Ajouter dans la cocotte.
			\item Quand toute la viande est dans la cocotte, déglacez la poêle avec du vin et faites bouillir en raclant pour récupérer les sucs. Salez, poivrez, ajoutez dans la cocotte.
			\item Ajouter dans la cocotte le bouquet garni, le chocolat, et recouvrir de vin. Ajouter du vin ou de l'eau si nécessaire. Salez, poivrez. Portez à ébullition.
			\item Une fois à petite ébullition, laissez mijoter à feu très doux pendant 4 heures (si possible en plusieurs fois).
		\end{enumerate}
	}

	\suggestion{%
		\begin{enumerate}
			\item Bien faire revenir la viande à feu fort pour la colorer (presque noire).
			\item Plus le plat aura mijoté doucement avec des phases de repos, meilleur ce sera.
		\end{enumerate}
	}
	\hint{%
		A servir avec un Bourgogne, évidemment ! Gevrey-Chambertin, Côtes de Nuits, Châteauneuf du Pape... Faites-vous plaisir.
	}

\end{recipe}
