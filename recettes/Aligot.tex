\begin{recipe}
[%
	preparationtime = {\unit[45]{min}}, %h, min, ...
	bakingtime={\unit[1]{h}}, %h, min, ...
	portion = {\portion{4}},
	source = {M. Viala}
]
{Aligot}

	%    \graph
	%    {% pictures
	%        small=pic/glass,     % small picture
	%        big=pic/ingredients  % big picture
	%     }

	\ingredients{%
		1kg de pommes de terre\\
		400g de tome fraîche\\
		20cl de crème fraîche\\
		150g de beurre\\
		4 gousses d'ail\\
		Poivre\\
		Sel\\
	}

	\preparation{%
		\begin{enumerate}
			\item Eplucher et couper les pommes de terre en dés.
			\item Faire cuire les pommes de terre dans l'eau froide (environ 30min).
			\item Pendant ce temps, émincer la tome fraîche en tranches fines.
			\item Eplucher et couper l'ail en morceaux grossiers.
			\item Quand les pommes de terre sont cuites, les passer en purée. Mettre dans une grande cocotte, à feu vif.
			\item Ajouter le beurre et la crème et mélangez bien.
			\item Ajouter progressivement la tome émincée, en mélangez constamment avec une grande spatule. Soulevez progressivement la pâte obtenue pour la faire filer.
			\item Ajoutez l'ail. Salez et poivrez à votre goût.
			\item Quand le mélange est bien filant, servez chaud !
		\end{enumerate}
	}

	\suggestion{%
		Servez bien chaud, accompagnez d'une salade verte pour un plat végétarien, ou bien d'une saucisse d'Auvergne pour l'expérience complète.
	}

	\suggestion[Variantes]
	{%
		N'hésitez pas à changer le fromage selon vos goûts, ou l'origine de la recette :
		Laguiole, Salers, Cantal... Faites votre choix !
	}
	\hint{%
		Avec un Marcillac, évidemment. Ou un Côtes d'Auvergne ou du Forez.
	}

\end{recipe}
