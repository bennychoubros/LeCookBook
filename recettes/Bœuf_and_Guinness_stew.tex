\begin{recipe}
[%
	preparationtime = {\unit[30]{min}}, %h, min, ...
	bakingtime={\unit[1,5]{h}}, %h, min, ...
	portion = {\portion{2-3}},
	source = {Inspiré par A. Lataillade}
]
{Bœuf and Guinness stew}

	%    \graph
	%    {% pictures
	%        small=pic/glass,     % small picture
	%        big=pic/ingredients  % big picture
	%     }

	\introduction{%
		Comme tout bon ragoût, l'idéal est de le préparer la veille et de le réchauffer à feu doux avant dégustation. 
	}

	\ingredients{%
		400g de boeuf (type paleron)\\
		200ml de Guinness\\
		1 carotte\\
		1 oignon\\
		1 tige de céleri\\
		40g de beurre\\
		1 litre de bouillon\\
		Thym\\
		Romarin
		Poivre\\
		Sel\\
	}

	\preparation{%
		\begin{enumerate}
			\item Couper le bœuf en cubes.
			\item Eplucher Les legumes. Couper la carotte en rondelles. Coupez l'oignon et la branche de céleri en dés.
			\item Faire fondre le beurre dans un cocotte chaude. Faites-y dorer les morceaux de bœuf à feu vif pendant 3 minutes pour les colorer.
			\item Ajouter carotte, céleri et oignons. Laisser cuire pendant 3 minutes en remuant bien.
			\item. Ajoutez, thym et romarin. Mouillez avec la Guiness et le bouillon.Salez. Poivrez.
			\item Couvrez partiellement, et laissez cuire pendant 1h30 à feu doux (petite ébullition).
		\end{enumerate}
	}

	\suggestion{%
		Comme suggéré par \texttt{A.Lataillade} sur son blog \texttt{PapillesetPupilles}\footnote{\url{ihttps://www.papillesetpupilles.fr/2012/11/irish-stew-ou-ragout-de-boeuf-irlandais-a-la-guinness.html/}}, c'est encore meilleur servi avec une purée de pomme de terre à la ciboulette (ou \texttt{Champ}). Ou avec la version originelle au chou, le \texttt{Colcannon}.
	}

\end{recipe}
