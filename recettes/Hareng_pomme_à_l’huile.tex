\begin{recipe}
[%
	preparationtime = {\unit[10]{min}}, %h, min, ...
	bakingtime={sans cuisson}, %h, min, ...
	portion = {\portion{2}},
	source = {T. Clouet}
]
{Hareng pomme à l’huile}

	%    \graph
	%    {% pictures
	%        small=pic/glass,     % small picture
	%        big=pic/ingredients  % big picture
	%     }

	\ingredients{%
		2 filets de harengs fumés\\
		4 pommes de terres nouvelles (type charlotte)\\
		1 échalote\\
		2 oignons nouveaux\\
		3 cuillères à soupe d'huile\\
		1 cuillère à soupe de vinaigre de vin rouge\\
		poivre\\
		sel\\
	}

	\preparation{%
		\begin{enumerate}
			\item Pelez les pommes de terre. Faites-les cuire dans une casserole remplie d’eau froide salée pendant une trentaine de minutes. Vérifiez la cuisson à l’aide de la pointe d’un couteau. Egouttez-les. Coupez-les en tranches.
			\item Otez la première peau de votre échalote et ciselez-la.
			\item Otez la première peau de vos oignons nouveaux puis coupez-les en rondelles. Séparez les anneaux.
			\item Dans un bol, mélangez le vinaigre avec du sel et du poivre puis ajoutez l’huile ainsi que l’échalote.
			\item Servez vos filets de harengs en les accompagnant de tranches de pommes de terre. Arrosez l’ensemble de vinaigrette. Agrémentez d’anneaux d’oignons nouveaux.
		\end{enumerate}
	}


	\hint{%
		Vous pouvez ajouter un peu de cerfeuil ou du persil plat à votre vinaigrette.
	}

\end{recipe}
