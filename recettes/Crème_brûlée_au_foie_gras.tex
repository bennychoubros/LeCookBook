\begin{recipe}
[%
	preparationtime = {\unit[20]{min}}, %h, min, ...
	bakingtime={\unit[30]{min}}, %h, min, ...
	bakingtemperature={\protect\bakingtemperature{
		topbottomheat=\unit[150]{\textcelsius}}
	},
	portion = {\portion{4}},
	source = {Bennychou}
]
{Crème brûlée au foie gras}

	%    \graph
	%    {% pictures
	%        small=pic/glass,     % small picture
	%        big=pic/ingredients  % big picture
	%     }

	\introduction{%
		Il est plus aisé de prévoir cette recette la veille, pour laisser la crème reposer au frigo suffisament longtemps.
	}

	\ingredients{%
		200g de foie gras mi-cuit\\
		20cl Lait\\
		20cl Crème entière Liquide\\
		4 jaunes d'œufs\\
		10cl de Muscat\\
		Sucre de canne roux, ou cassonade\\
		1 cuillère à café de muscade\\
		poivre\\
		sel\\
	}

	\preparation{%
		\begin{enumerate}
			\item Préchauffez le four à 150{\textcelsius}. 
			\item Coupez le foie gras en morceaux grossiers.
			\item Mixez le foie gras avec le muscat, la crème, le lait, les jaunes d’œufs, sel, poivre et muscade. Mélangez jusqu'à ce que ce soit parfaitement homogène. 
			\item Versez dans les ramequins et cuire 30 min au four.
			\item Laissez refroidir. Les réservez au frigo jusqu'au lendemain. 
			\item Juste avant de servir, saupoudrez de sucre et faites caraméliser les crèmes au chalumeau. 
		\end{enumerate}
	}

	\suggestion[Hors-d'Œuvre]
	{%
		Recette parfaite pour une entrée ou un dessert pour les adeptes du salé, c'est auussi une très bonne mise en bouche. N'hésitez pas à préparer des petites portions individuelles, dans des cuillères allant au four ou des mini-ramequins.
	}

	\hint{%
		S'accompagne bien d'un vin blanc moelleux : Gewurtzraminer ou Coteaux du Layon par exemple. Et pourquoi pas essayer un Saint Georges - Saint-Emilion si c'est le plat principal. 
	}

\end{recipe}
